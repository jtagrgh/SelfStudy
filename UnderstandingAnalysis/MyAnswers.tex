\documentclass{article}
\usepackage{graphicx} % Required for inserting images
\usepackage{amsthm}
\usepackage{amsmath}
\usepackage{amsfonts}
\usepackage{amssymb}
\usepackage{hyperref}
\usepackage{enumerate}
\usepackage{xfrac}

\newtheorem{theorem}{Theorem}[section]
\newtheorem{lemma}[theorem]{Lemma}
\newcommand{\inv}{^{-1}}
\newcommand{\N}{\mathbb{N}}
\newcommand{\R}{\mathbb{R}}
\newcommand{\Q}{\mathbb{Q}}
\newcommand{\Z}{\mathbb{Z}}
\newcommand{\I}{\mathbb{I}}
\newcommand{\nth}{\frac{1}{n}}

\title{Understanding Analysis Exercise Answers}
\author{ Jakob Koblinsky }
\date{November 2024}

\begin{document}

\maketitle

\section{The Real Numbers}
\setcounter{subsection}{1}
\subsection{Some Preliminaries}
\begin{theorem}
    If $m|n$ for $m$ and $n\in \mathbb{Z}$, then $m|n^2$.
\end{theorem}
\begin{proof}
    If $m$ divides $n$ if and only if there exists an integer $k$ such that $n=km$. Therefore, $n^2=(km)^2=k(km^2)$.
\end{proof}

\begin{theorem}
\label{sqr_div}
    If $m \nmid q^2$ for any $q \in \{ 1,2,\dots,m-1 \}$, then $m|n^2$ implies $m|n$.
\end{theorem}
\begin{proof}
    $m\nmid n$ if and only if there exists $k\in \mathbb{Z}$ and $q\in\{1,\dots,m-1\}$ such that $n=km+q$. Therefore, $n^2=(mk)^2+2qmk+q^2$, which after factoring is $mp+q^2$ where $p\in\mathbb{Z}$. Thus, $m|n^2$ only if $m|q^2$, but we assumed the negation, so $m \nmid n^2$.
\end{proof}

\paragraph{Exercise 1.2.1.} 
\begin{enumerate}[(a)]
    \item Assume for the sake of contradiction that $\sqrt{3}$ is rational. Then there exists co-prime integers $p$ and $q$ such that $(p/q)^2=3$. Thus $p^2=3q^2$, and since 3 divides neither $1^2$ nor $2^2$, by \autoref{sqr_div}, $p=3k$ for some integer k. Therefore $q^2=3k^2$ and so $q=3r$ for some integer $r$. That would mean $p$ and $q$ are not co-prime, which is a contradiction. Moreover, 6 $\nmid q^2$ for any $q\in \{1,2,3,4,5\}$, and so a similar argument can be made. 
    \item In trying to use a similar argument, we will fail to infer that $q$ is also even. In detail, if $(p/q)^2=4$ then $p^2=4q^2$ and so $p=2k$ for an integer $k$. Thus $4k^2=4q^2$, and so $k^2=q^2$, which fails to provide any information about the parity of $q$.
\end{enumerate}

\begin{enumerate}[(a)]
    \item If 
\end{enumerate}

\begin{theorem}
\label{even_pow}
    An integer $p$ is even if and only if $p^n$ is even for any natural $n$.
\end{theorem}
\begin{proof}
    ($\implies$) If $p$ is even, then $p=2k$ for some integer $k$, and so $p^2=(2k)^2=2(2k^2)=2r$ where $r$ is an integer. Thus $p^2$ is even. Furthermore, if $p^n$ is even then $p^{n+1}=p^np$ which is the product of even integers, and thus even. ($\impliedby$) Conversely, if $p$ is odd, then $p=2k+1$ for some integer $k$, and so $p^2=2(2k^2+2k)+1=2r+1$ for some integer $r$. Thus $p^2$ is odd. Furthermore, if $p^n$ is odd, then $p^{n+1}=p^np$ which is the product of odd integers, and thus odd.
\end{proof}

\paragraph{Exercise 1.2.2.} Assume for the sake of contradiction that there exists some rational $r$ satisfying $2^r=3$. Then there exists integers $p$ and $q$ such that $r=p/q$, and so $2^p=3^q$. We know $r$ cannot be 0, because $1 \neq 3$, and so $p$ and $q$ are not 0. Also $p$ and $q$ cannot be negative, because if $p$ were negative then $r$ would be negative, and $1/2^r \neq 3$ for any natural $r$. Therefore $p$ and $q$ must be natural. By \autoref{even_pow}, we know $2^p$ is even and $3^q$ is odd. An even number is never odd, and so we have a contradiction.

\paragraph{Exercise 1.2.3.} 
\begin{enumerate}[(a)]
    \item False. Let $A_n=\{n,n+1,\dots\}$. Assume for the sake of contradiction that there exists some natural $a$ in $\bigcap_{n=1}^\infty A_n$. Then $a$ must be in every $\{A_n\}_{n=1}^\infty$, but $a \notin A_{a+1}$. Therefore $\bigcap_{n=1}^\infty A_n = \emptyset$.
    \item True.
    \item False. $\{1\} \cap (\{1\}\cup\{1,2\}) = \{1\} \neq \{1,2\}=(\{1\}\cap\{1\})\cup\{1,2\}.$
    \item True.
    \item True.
\end{enumerate}

\paragraph{Exercise 1.2.4.}
To construct $A_n$, we may lay the natural number out diagonally as follows.

\begin{tabular}{ccccc}
    $A_1$ & $A_2$ & $A_3$ & $A_4$ & $\dots$ \\
    \hline
     1 & 3 & 6 & $\vdots$ & \\
     2 & 5 & $\vdots$ \\
     4 & $\vdots$ \\
     \vdots
\end{tabular}

Here we have an infinite set $A_n$ for every $n\in\mathbb{N}$. Moreover, each natural number appears in exactly one set, so $A_i\cap A_j=\emptyset$ for $i\neq j$ and $\bigcup_{i=1}^\infty A_i = \mathbb{N}$.

\paragraph{Exercise 1.2.5 (De Morgan’s Laws).} We will assume De Morgan's law for predicate logic.
\begin{enumerate}[(a)]
    \item $x\in(A\cap B)^c \iff \neg(x\in A\cap B) \iff \neg(x\in A \text{ and } x \in B) \iff \neg(x\in A) \text{ or } \neg(x\in B) \iff x\in A^c \text{ or } x\in B^c \iff x \in A^c \cup B^c$.
    \item The inverse follows from the above, and so $(A\cap B)^c = A^c \cup B^c$.
    \item $x\in (A\cup B)^c \iff \neg(x\in A \text{ or } x\in B) \iff x\in A^c \text{ and } x\in B^c\iff x\in A^c\cap B^c$.
\end{enumerate}

\paragraph{Exercise 1.2.6.}
\begin{enumerate}[(a)]
    \item  If $a$ is positive then $a=|a|$. If $a$ and $b$ are positive, then $a+b$ is positive, so $|a+b|=a+b=|a|+|b|$.
    
    If $a$ is negative then $-a=|a|$. If $a$ and $b$ are both negative, then $a+b$ is negative, so $|a+b|=-(a+b)=(-a)+(-b)=|a|+|b|$.

    \item $(a+b)^2=a^2+2ab+b^2$ and $(|a|+|b|)^2=|a|^2+2|a||b|+|b|^2=a^2+2|a||b|+b^2$, because $|a|^2=|a^2|=a^2$. Therefore $(a+b)^2 \leq (|a|+| b|)^2$ if and only if $ab\leq|a||b|$. We may re-arrange that inequality as $\pm a\leq |a|$, which is always the case. Finally, if $x$ and $y$ are positive, $x\leq y$ if and only if $x^2\leq y^2$, and so $|a+b|\leq|a|+|b|$.

    \item Assuming the triangle inequality, $|a+b+c|=|a+t|\leq|a|+|t|=|a|+|b+c|\leq|a|+|b|+|c|$. Therefore $|a-b|=|(a-c)+(c-d)+(d-b)|\leq |a-c|+|c-d|+|d-b|$.

    \item Let $|a|\geq |b|$. Then $|a|-|b|=\epsilon$, for some $\varepsilon \geq0$. Therefore $||a|-|b||=|\varepsilon|=\varepsilon=|a|-|b|$. By the unremarkable identity, that is equivalent to $|(a-b)+b|-|b|$, which by the Triangle Inequality is $\leq |a-b|$.
\end{enumerate}

\begin{theorem}
\label{sub_range}
    If $A \subseteq B$ then $g(A) \subseteq g(B)$.
\end{theorem}
\begin{proof}
    $y\in g(A)$ if and only if $\exists x \in A$ such that $g(x)=y$. Furthermore, since $x\in A$, then $x\in B$, and so $\exists x \in B$ such that $g(x)=y$. Therefore $y\in g(B)$.
\end{proof}

\paragraph{Exercise 1.2.7.}
\begin{enumerate}[(a)]
    \item $f(A\cap B)=[1,4]=f(A)\cap f(B)$. And $f(A\cup B)=[0,16]=f(A)\cup f(B)$.
    \item Let $A=[-2,-1]$ and let $B=[1,2]$. Then $f(A\cap B)=\emptyset \neq [1,4] = f(A)\cap f(B)$.
    \item If $y\in g(A\cap B)$ then, by \autoref{sub_range}, $y\in g(A)$ and $y\in g(B)$ and so $y\in g(A) \cap g(B)$.
    \item \textit{Conjecture.} $g(A\cup B) = g(A) \cup g(B)$. \textit{Proof.} If $y\in g(A) \cup g(B)$ then $y\in g(A)$ or $y\in g(B)$. If $y\in g(A)$ then $y\in g(A\cup B)$, because $A\subseteq A\cup B$. Likewise for $y\in g(B)$. Conversely, if $y\in g(A\cup B)$, then $\exists x \in A$ or $\exists x \in B$ such that $g(x)=y$. If $x\in A$ then $y \in g(A)$ whereby for free we get that $y\in g(A) \cup g(B)$. Likewise for $x \in B$.
\end{enumerate}

\paragraph{Exercise 1.2.8.}
\begin{enumerate}[(a)]
    \item $f(n)=2n$.
    \item $f(n) = n/2$ if $n$ is even, otherwise $(n+1)/2$ if $n$ is odd.
    \item $f(n) = n/2$ if $n$ is even, otherwise $-(n-1)/2$ if $n$ is odd.
\end{enumerate}

\begin{theorem}
    For any $x$ in the domain of $f$, $x\in f^{-1}(A)$ if and only if $x\in A$.
\end{theorem}
\begin{proof}
    By definition $x\in f^{-1}(A)$ if and only if $x\in \{x\in D: f(x) \in A\}$, therefore $f(x)\in A$.
\end{proof}

\paragraph{Exercise 1.2.9.}
\begin{enumerate}[(a)]
    \item $f^{-1}(A) = [0,2]$ and $f^{-1}(B) = [0,1]$. So $f^{-1}(A\cap B) = [0,1] = f^{-1}(A) \cap f^{-1}(B)$. And $f^{-1}(A\cup B) = [0,2] = f^{-1}(A) \cup f^{-1}(B)$.
    \item $x\in g\inv(A\cap B) \iff g(x) \in A\cap B \iff g(x) \in A$ and $g(x) \in B$ $\iff x\in g\inv(A)$ and $x \in g\inv(B)$ $\iff x\in g\inv(A)\cap g\inv(B)$.

    $x\in g\inv(A\cup B) \iff g(x)\in A$ or $g(x) \in B$ $\iff x\in g\inv(A)$ or $x\in g\inv(B)$ $\iff x\in g\inv(A) \cup g\inv(B)$.
\end{enumerate}

\paragraph{Exercise 1.2.10.}
\begin{enumerate}[(a)]
    \item \textit{Counterexample.} If $a=b$ then $a-b=0$ and $\varepsilon > 0$ so $a-b < \varepsilon$. Therefore $a<b+\varepsilon$. So the posterior condition holds, but $a=b$ so clearly the anterior condition does not hold.
    \item \textit{Proof.} Since $\varepsilon > 0$, then $\varepsilon + b > b$. Therefore, if $a<b$, then $a<b<b+\varepsilon$.
    \item \textit{Proof.} If $a\leq b$ then $a\leq b < b+\varepsilon$ for $\varepsilon > 0$. Conversely, let $a<b+\varepsilon$ for $\varepsilon > 0$, and assume for the sake of contradiction that $a > b$. Because $a > b$ then $a-b$ is some positive $\varepsilon_0$. But we know $a<b+\varepsilon$ for any positive $\varepsilon$. so surely, $a-b<\varepsilon_0$, but that is a contradiction to the fact that $a-b=\varepsilon_0$. Therefore it must be that $a\leq b$.
\end{enumerate}

\paragraph{Exercise 1.2.11.}
\begin{enumerate}[(a)]
    \item There exists real numbers satisfying $a<b$ where $(a+1)/n \geq b$ for all $n\in \mathbb{N}$. I will guess that the claim is true.
    \item For all real numbers $x>0$, there exists some $n\in \mathbb{N}$ such that $x\geq 1/n$. I will guess that the negation is true.
    \item There exists two distinct real numbers that has no rational number between them.
\end{enumerate}

\paragraph{Exercise 1.2.12.}
\begin{enumerate}[(a)]
    \item $y_1=6>-6$, and $y_n>-6 \implies 2y_n > =-12 \implies 2y_n -6 > -18 \implies (2y_n-6)/3 > -6 \implies y_{n+1} > -6$.
    \item $y_1=6 > 2 = (2y_1-6)/3=y_2$, and $y_n>y_{n+1} \implies (2y_n-6)/3 > (2y_{n+1}-6)/3 \implies y_{n+1} > y_{n+2}$.
\end{enumerate}

\newcommand{\infcupset}{\bigcup_{i=1}^\infty A_i}
\newcommand{\infcapset}{\bigcap_{i=1}^\infty A_i}

\paragraph{Exercise 1.2.13.}
\begin{enumerate}[(a)]
    \item $(A_1)^c=A_1^c$, and $(A_1\cup A_2)^c=A_1^c \cap A_2^c$ as shown in Exercise 1.2.5. Finally $(\bigcup_{i=1}^nA_i)^c = \bigcap_{i=1}^n A_i^c \implies \bigcap_{i=1}^{n+1} A_i^c = (\bigcup_{i=1}^nA_i)^c \cap A_{n+1}^c = (\bigcup_{i=1}^nA_i \cup A_{n+1})^c = (\bigcup_{i=1}^{n+1}A_i)^c$.
    \item Let $B_i=\{i, i+1,...\}$ for any $i \in \N$. Then $\bigcap_{i=1}^\infty B_i = \emptyset$, but $n \in \bigcap_{i=1}^{n}B_i$.
    
    \item ($\subseteq$) $(\infcupset)^c \subseteq \infcapset^c$ if and only if $ x\notin (\infcupset)^c$ or $x\in \infcapset^c$, which is the case when $x\in(\infcupset) \cup (\infcapset^c) $, and so, using the infinite distributive law, $ x\in \bigcap_{j=1}^\infty(\infcupset \cup A_j^c) $. Note that $\infcupset \cup A_j^c = \ldots \cup A_j \cup A_j^c \cup \ldots = \mathcal{U}$. So, $ x \in \mathcal{U}$, which is always the case.

    ($\supseteq$) Similar to before, $(\infcupset)^c \supseteq \infcapset^c$ if and only if $x\notin \infcapset^c$ or $x\in (\infcupset)^c$, which is the case when $x \notin \infcapset^c \cap \infcupset$. Moreover, by the infinite distributive law, we know $x\notin \bigcup_{j=1}^\infty(\infcapset^c \cap A_j)$. Note that $\infcapset^c \cap A_j = \ldots \cap A_j^c \cap A_j \cap \ldots = \emptyset$. Therefore $x\notin \emptyset$, which is always the case.
\end{enumerate}


\subsection{The Axiom of Completeness}
\paragraph{Exercise 1.3.1.}
\begin{enumerate}[(a)]
    \item A real number $i$ is the \textit{greatest lower bound} for a set $A \subseteq R$ if it meets the following two criteria: (i) $i$ is a lower bound for A; (ii) if $l$ is any lower bound for $A$ then $l \leq i$.
    \item \textit{Lemma.} Let $l \in \R$ be a lower bound for $A \subseteq R$. Then $l = \inf A$ if and only if for every $\varepsilon > 0$, there exists some $a\in A$ such that $a < l + \varepsilon$. \textit{Proof.} If $l=\inf A$ then $l + \varepsilon > l$ and thus is not a lower bound, consequently it is not the case that $l + \varepsilon \leq a$ for every $a \in A$. In other words, there is some $a \in A$ such that $l + \varepsilon > a$. Conversely, if $l$ is a lower bound, and every value greater than $l$ given by $l + \varepsilon$ is no longer a lower bound, then all lower bounds must be less than or equal to $l$. Therefore $l = \inf A$.
\end{enumerate}

\paragraph{Exercise 1.3.2.}
\begin{enumerate}[(a)]
    \item \textit{Example.} $B = \{1\}.$ The max and min of $B$ is $1$, so the $\inf$ and $\sup$ of $B$ is 1. Therefore $\inf B \geq \sup B$.
    \item \textit{Impossible.} If set $B$ contains it's infimum then it is clearly non-empty. Additionally, since $B$ is finite, it has a maximum, which is it's supremum. Therefore, it must contain it's supremum.
    \item \textit{Example.} Let $B = \{ r \in Q : 1 < r \leq 2 \}$. We have $\sup B = 2$, which is in $B$, but $\inf B = 1$, which is not in $B$.
\end{enumerate}

\paragraph{Exercise 1.3.3.}
\begin{enumerate}[(a)]
    \item Since $A$ is bounded below, $B$ is non-empty. Moreover, $\inf A \in B$ and $\inf A \geq b$ for any $b \in B$. Therefore, $\inf A$ is also the maximum of $B$, and so $\sup B = \inf A$.
    \item For any non-empty $A \subseteq \R$ that is bounded below, let $B$ be the set of it's lower bounds. We know $B$ is non-empty because $A$ is bounded below, and that it is bounded above by any $a \in A$. Therefore, by the Axiom of Completeness, $\sup B \in \R$, and finally, by part (a), we know that $\inf A \in \R$.
\end{enumerate}

\paragraph{Exercise 1.3.4.}
\begin{enumerate}[(a)]
    \item Let $\alpha = \max\{\sup A_1, \sup A_2\}$, and let $a \in A_1 \cup A_2$, so that $a \in A_1$ or $a \in A_2$. If $a \in A_1$, we know $\alpha \geq \sup A_1$, and so $\alpha \geq a$. The same can be said for $a \in A_2$. Therefore, for $\alpha$ is an upper bound for $A_1 \cup A_2$. Additionally, for any $\varepsilon > 0$, $\alpha - \epsilon$ is less than $\sup A_1$ or $\sup A_2$. If $\alpha - \varepsilon < \sup A_1$, then there exists $a \in A_1$ such that $\alpha - \varepsilon < a$. Moreover, $a$ must but be in $A_1 \cup A_2$. The same can be said for $A_2$. Therefore $\alpha = \sup(A_1 \cup A_2)$.

    So now we know that $\max\{\sup A_1\} = \sup A_1$, and $\max\{\sup A_1, \sup A_2\} = \sup(A_1 \cup A_2)$. Assume that $\sup(\bigcup_{k=1}^n A_k) = \max\{\sup A_k\}_{k=1}^n$. Then, $\sup(\bigcup_{k=1}^{n+1} A_k) = \sup(\bigcup_{k=1}^{n} A_k \cup A_{n+1}) = \max\{\sup \bigcup_{k=1}^{n} A_k, \sup A_{n+1}\} = \max\{\max\{\sup A_k\}_{k=1}^n, \sup A_{n+1}\} = \max\{\sup A_k\}_{k=1}^{n+1}$. And so the formula is valid for any finite $n$.

    \item The formula does not always hold in the infinite case. If $\sup(\bigcup_{k=1}^\infty A_k)$ is not bounded above, then it can of course have no least upper bound. Let $A_k = \{ k \}$, so that $A_k$ is non-empty and bounded above by $k$. Now, for any $b$, we know $b+1 \in A_{b+1}$, and so $b+1 \in \bigcup_{k=1}^\infty A_k$. Therefore, $\bigcup_{k=1}^\infty A_k$ is not bounded above.
\end{enumerate}

\paragraph{Exercise 1.3.5.}
\begin{enumerate}[(a)]
    \item (i) For any $a \in A$, by definition $\sup(A) \geq a$, and so, for $c \geq 0$, we also have    $c\sup(A) \geq ca$. Therefore, $c\sup(A)$ is an upper bound of $cA$. (ii) For any $\varepsilon > 0$, there is some $a \in A$ such that $\sup(A)-\varepsilon < a$, and likewise, $c\sup(A) - \varepsilon < ca$. Therefore, $c\sup(A)$ is the supremum of $cA$.
    \item (i) For any $a\in A$, by definition $\inf(A) \leq a$, and so, for $c < 0$, we also have $c\inf(A) \geq ca$. Therefore $c\inf(A)$ is an upper bound of $cA$. (ii) Let $l$ be any upper bound of $cA$, that is, $b \geq ca$ for any $a \in A$. Equivalently, $b/c \leq a$, and so, $b/c \leq \inf(A)$. Multiplying through by $c$, we get $b \geq c\inf(A)$. Therefore, $c\inf(A)$ is the supremum of $cA$ when $c<0$.
\end{enumerate}

\paragraph{Exercise 1.3.6.}
\begin{enumerate}[(a)]
    \item By definition $s \geq a$ for any $a \in A$, and $t \geq b$ for any $b \in b$. Therefore, $s + t \geq a + b$ for any $a$ and $b$. So, $s+t$ is an upper bound for $A + B$.
    \item Let $u$ be an arbitrary upper bound for $A+B$, then by definition $u \geq a+b$ for any $a\in A$ and $b\in B$. Equivalently, $u-a \geq b$, and so $u-a$ is an upper bound for $B$. Therefore, $u-a \geq t$ must hold.
    \item Following from above, we have $u-a\geq t$ for an arbitrary upper-bound of $A+B$ given by $u$, and any $a \in A$. Equivalently, $u-t\geq a$, so $u-t$ is an upper bound for $A$, and necessarily, $u-t\geq s$. Equivalently, $u\geq t+s$, and therefore $s+t$ is the supremum of $A+B$.
    \item By Lemma 1.3.8, every any $\varepsilon > 0$, there exists $a\in A$ and $b \in B$ such that $s-\varepsilon < a$ and $t-\varepsilon < b$. Therefore, $s+t-\varepsilon < a+b$, and by the converse of Lemma 1.3.8, $s+t$ is the supremum of $A+B$.
\end{enumerate}

\paragraph{Exercise 1.3.7.}
Let $a$ be an upper bound for $A$ and also an element of $A$. (ii) Any other upper bound can be no less than $a$, because $a \in A$, so $a$ is the supremum.

\paragraph{Exercise 1.3.8.}
\begin{enumerate}[(a)]
    \item $\sup = 1$, $\inf = 0$.
    \item $\sup=1$, $\inf=-1$.
    \item $\sup=1/3$, $\inf=1/4$.
    \item $\sup=1$, $\inf = 0$.
\end{enumerate}


\paragraph{Exercise 1.3.9.}

\begin{enumerate}[(a)]
    \item By Lemma 1.3.8, we know that there exists $b \in B$ satisfying $\sup(B) - \varepsilon < b$ for every $\varepsilon > 0$. Let $b$ satisfy the case when $\varepsilon = \sup(B) - \sup(A)$. That means $\sup(A) < b$, and so $b$ is an upper bound for $A$.
    \item Let $A=\{1\}$ and let $B=\{n/(n+1): n\in \N\}$. We have $\sup(A)=1=\sup(B)$, but $1 > b$ for any $b\in B$, and therefore there is no $b\in B $ that is an upper bound for $A$.
\end{enumerate}

\paragraph{Exercise 1.3.10. (Cut Property)}
\begin{enumerate}[(a)]
    \item Let $c=\sup(A)$. Since $A$ is non-empty, by the Axiom of Completeness, $\sup(A) \in \R$. Because $A\cup B = \R$, and $A \cap B = \emptyset$, every real number must be in exclusively one or the other. So, if $\sup(A) \in A$, then we know $\sup(A) < b$ for every $b \in B$. Otherwise, if $\sup(A) \in B$, then $\sup(A) = \inf(B)$, because, for any $\varepsilon > 0$, we have $\sup(A) + \varepsilon > \sup(A)$. So, since $\sup(A) = \inf(B)$, it must be that $\sup(A) \leq b$ for every $b\in B$. Additionally, by definition, $\sup(A) \geq a$ for every $a\in A$.
    \item Let $E$ be a non-empty set that is bounded above, and let $A$ be the set of reals where $a > x$ for every $a\in A$ and $x\in E$. Since $E$ is bounded above, there exists some $y \in \R$ such that $y \geq x$ for every $x \in E$. Therefore, $y+1 > x$, and so $A$ is non-empty. Then, by the Cut Property, there exists $c \in \R$ such that $x \leq c \leq a$ for every $x \in E$ and $a \in A$. If $c\in E$, then $c$ is the maximum of $E$ and thus $\sup(E)=c$. If $c\notin E$ then $c\in A$, and so $c$ is the minimum of $A$. Then, for any $\varepsilon > 0$, we may construct $x = c-\varepsilon/2$, which must be in $E$ as it is less than $\min(A)$, to satisfy $c - \varepsilon < x$. Therefore, $\sup(E) = c$.
    \item Let $A = \{r \in \Q : r^2 < 2 \text{ or } r < 0 \}$ and let $B = \{r \in \Q : r^2 \geq 2 \text { and } r \geq 0\}$. We have $A \cup B = \Q$, and $A \cap B = \emptyset$. Additionally, $a < b$ for any $a \in A$ and $b \in B$. There is only one $c$ that satisfies, $a \leq c \leq b$. That is, when $c^2 = 2$, but then $c \notin \Q$.
\end{enumerate}

\paragraph{Exercise 1.3.11.}
\begin{enumerate}[(a)]
    \item \textit{True.} $\sup(B) \geq b$ for every $b \in B$, and any $a \in A$ is also in $B$, so $\sup(B) \geq a$ for any $a \in A$. Therefore, $\sup(B)$ is an upper bound for $A$, and so $\sup(B) \geq \sup(A)$.
    \item \textit{True.} If $\sup(A) < \inf (B)$ then $\inf(B) - \sup(A) > 0$, so let $\inf(B) - \sup(A) = \varepsilon_0$. Let $c = \sup(A) + \varepsilon_0/2$. Then, $c < \sup(A)+\varepsilon_0 = \inf(B) \leq b$ for any $b \in B$. Also, $c > \sup(A) \geq a$ for any $a \in A$. Thus, $a < c< b$ for all $a \in A$ and $b \in B$.
    \item \textit{False.} Let $A = [0,2)$ and $B = (2, 4]$. Then, for $c =2$, it follows that $a<c<b$ for every $a\in A$ and $b \in B$. But, $\sup(A) = 2 = \inf(B)$, and the converse of (b) does not hold.
\end{enumerate}


\subsection{Consequences of Completeness}
\paragraph{Exercise 1.4.1.}
\begin{enumerate}[(a)]
    \item Let $a$ and $b$ be rational numbers, so that $a=p/m$ and $b=q/n$ for $p,q\in \Z$ and $m,n\in \N$. Then, $ab=pq/mn$, which is a rational because $pq\in \Z$ and $mn \in \N$. Additionally, $a+b=(pn+bm)/mn$ which is rational. Therefore, $\Q$ is closed under addition and multiplication.
    \item Let $a\in \Q$ and $t\in \I$. Then, if $a+t$ were some rational $b$, that would mean $t=b-a$, which would be a contradiction as $b-a \in \Q$. Also, if $at$ were some rational $b$, that would mean $t=b/a$, which would be a contradiction as $b/a \in \Q$ for $a \neq 0$. Therefore, $a+t\in I$ and $at \in I$.
    \item $\sqrt{2}\times \sqrt{2} = 2$ which is not in $\I$, so $\I$ is not closed under multiplication. Also, $\sqrt{2} + (1 - \sqrt{2}) = 1$, which is not in $\I$, and thus $\I$ is not closed under addition.
\end{enumerate}

\paragraph{Exercise 1.4.2.}
First, to show that $s$ is an upper bound for $A$. By the Archimedean Property, for any $\varepsilon > 0$, there exists $n\in \N$ such that $1/n < \varepsilon$, and so $s+\varepsilon > 1/n > a$ for any $a \in A$. Therefore, $s\geq a$ for any $a\in A$, making $s$ an upper-bound. Now, to show that $s$ is the least upper-bound. Similar to before, for any $\varepsilon$ there exists $n$ such that $s-\varepsilon < s-1/n < a$ for some $a \in A$. Therefore, $s$ is the supremum.

\paragraph{Exercise 1.4.3.}
$0 \notin (0,1)$ so $0\notin \bigcap_{n=1}^\infty (0,1/n)$. Additionally, for any $\varepsilon > 0$ there exists $n\in \N$ such that $1/n < \varepsilon$, and so $\varepsilon \notin (0,1/n)$. Therefore $\bigcap_{n=1}^\infty (0,1/n)$ is empty.

\paragraph{Exercise 1.4.4.}
Since $T \subseteq [a,b]$ and $\sup[a,b] = b$, it must be that $b$ is an upper-bound for $T$. Additionally, for any $\varepsilon > 0$, there exists $c\in [a,b]$ such that $b-\varepsilon < c$. And since $\Q$ is dense in $\R$, there exists $r\in \Q$ such that $b-\varepsilon < r < c$. Therefore $\sup T = b$.

\paragraph{Exercise 1.4.5.}
For any reals $a<b$, we know $a+\sqrt{2}<b+\sqrt{2}$ are also reals, and so, there exists $r\in \Q$ satisfying $a+\sqrt{2} < r < b+\sqrt{2}$. Therefore, $a < r-\sqrt{2} < b$, where $r-\sqrt{2}$ is certainly irrational. Therefore $\I$ is dense in $\R$.

\paragraph{Exercise 1.4.6.}
\begin{enumerate}[(a)]
    \item Let $a=0$, and $b=1/11$. By inspection, there is no $p\in \Z$ and $q\in \N$ satisfying $q \leq 10$ and $0<p/q<1/11$. Therefore rationals of that form are not dense in $\R$.
    \item Let $a<b$ be real numbers. Then there exists $n \in N$ satisfying $1/n < b-a$ and so $1/2^n < 1/n < b-a$. Equivalently $b-1/2^n > a$. Now let integer $m = 2^na+ 1$, so that $m>2^na\geq m-1$. Immediately we get that $m/2^n>a$. Additionally, $2^n(b-1/2^n) > 2^na \geq m-1$ and so $b>m/2^n$. Therefore, there exists $m\in \Z$ and $n\in \N$ satisfying $a<m/2^n<b$.
    \item The smallest positive rational satisfying $10|p|\geq q$ is $p=q/10$. In that case $p/q=1/10$. Therefore, there is no such $p/q$ satisfying $0<p/q<1/11$ and so they are not dense in $\R$.
\end{enumerate}

\paragraph{Exercise 1.4.7.}
Assume for the sake of contradiction that $\alpha= \sup T$ satisfies $a^2>2$. Clearly $\alpha > \alpha - 1/n$. Now to show that $a-1/n$ remains an upper-bound, that is, $(\alpha-1/n)^2 \geq 2$. Expanding the left side of the inequality, we get $(\alpha-1/n)^2 = \alpha^2 - 2\alpha /n + 1/n^2 > \alpha^2 - 2\alpha/n$. Now, let $n$ satisfy $n > 2\alpha / (\alpha^2 - 2)$ so that $2\alpha /n < \alpha^2 -2$. Thus, $\alpha^2 - 2\alpha /n > \alpha^2 - (\alpha^2 -2) = 2$. Therefore if $\alpha^2 > 2$ it can not be the supremum.

\paragraph{Exercise 1.4.8.}
\begin{enumerate}
    \item \textit{Example.} Let $A = (2n-1)/2n = \{ \frac{1}{2}, \frac{3}{4}, \frac{5}{6}, \ldots \}$ and $B = 2n/(2n+1) = \{ \frac{2}{3}, \frac{4}{5}, \frac{6}{7}, \ldots\}$. These sets satisfy $A \cap B = \emptyset$, $\sup A = \sup B = 1$, and $1\notin A \cup B$.
    \item \textit{Example.} Let $J_n = (2-\nth, 2+\nth) = \{ x \in \R : 2-\nth < x < 2+\nth \}$. Then for any $n$, $2-1/n < 2-1/(n+1)$ and $2+1/(n+1) < 2+1/n$. Therefore $J_n \subseteq J_{n+1}$. Now let $A = \{ 2-1/n : n\in N\}$ or the set of left-hand endpoints, and $B = \{ 2+1/n: n\in N\}$ or the set of right-hand endpoints. Then $\sup A = 2$ and $\inf B = 2$ but $2\notin A$ and $2\notin B$ so  $a_n < 2 < b_n$ for all $n$. Therefore $2\in J_n$ for any $n$ and so $2\in \bigcap_{n=1}^\infty J_n$. Now to show that $\bigcap_{n=1}^\infty J_n$. Assume for the sake of contradiction that there is some $c \neq 2$ in  $\bigcap_{n=1}^\infty J_n$. We may not have $c > 2$ because $c\notin J_n$ for $1/n < c-2$. Moreover it can not be that $c < 2$ because $c\notin J_n$ for $1/n>2-c$. Therefore $\bigcap_{n=1}^\infty$ is finite as it contains only 2.
    \item \textit{Example.} Let $L_n = \N \cap [n,\infty) = \{ n, n+1, \ldots \}$. Then any $n \notin L_{n+1}$ and so $\bigcap_{n=1}^\infty L_n = \emptyset$.
    \item \textit{Impossible.} Let $I_n = [a_n, b_n]$ and $J_n = \bigcap_{k=1}^n I_k$. 
    Then for any $x \in J_n$ it must be that $x \geq a$ for all $a \in \{a_1, \ldots, a_n\}$ and $x\leq b$ for all $b \in \{b_1, \ldots, b_n\}$. By our assumption that $\bigcap_{k=1}^n I_k = J_n$ is non empty for every $n\in \N$, we must conclude that $J_n$ is a valid closed interval. Additionally, since $\bigcap_{k=1}^n I_k \supseteq \bigcap_{k=1}^{n+1} I_k$ it follows that $J_n \supseteq J_{n+1}$ for any $n \in \N$. Finally, by the Nested Interval Property, $\bigcap_{n=1} ^\infty I_n = \bigcap_{n=1}^\infty J_n \neq \emptyset$.
\end{enumerate}




\subsection{Cardinality}
\paragraph{Exercise 1.5.1.}
Let $n_1 = \min\{n\in \N : f(n) \in A\}$, and in general, let $n_k = \min\{ n\in \N : f(n) \in A, n > n_{k-1}\}$. That means $n_k$ is the $k^{th}$ smallest natural number in $f\inv(A)$. Surely each $n_k$ is unique because the sequence is strictly increasing. Also, let $n \in f\inv (A)$, surely $\{ m \in f\inv(A) : m < n\}$ is finite, so $n$ is greater than some number of elements in $f\inv (A)$. Let that number be $k$, then $a_k = n$. We may use this correspondence of natural number to define $g:\N \to A$, where $g(k) = f(n_k)$ for every $k\in N$.

\paragraph{Exercise 1.5.2.}
The Nested Interval Property asserts the existence of some $x\in \R$ in the infinite intersection of nested closed intervals. So the error in the proof is that the element in $\bigcap_{n=1}^\infty I_n$ is not necessarily rational, and so there is no certainty of a contradiction.

\paragraph{Exercise 1.5.3.}
\begin{enumerate}[(a)]
    \item Let $A_1$ and $A_2$ be countable sets, and let $B_2 = A_2 \backslash A_1$. Since $A_1$ countable, there exists a 1-1 correspondence $f: \N \to A_1$. Also, since $B_2 \subseteq A_2$, either $B_2$ is countable or finite. If $B_2$ is countable, then there exists a 1-1 correspondence $g: \N \to B_2$. Then, we may define the function $g:\N \to A_1 \cup B_2$ given by $h = \begin{cases}f((n+1)/2) & \text{ if n is odd }\\ g(n/2) & \text{ if n is even }\end{cases}$. Since $A_1$ and $B_2$ are disjoint, for every $n\in \N$, $g(n)$ is in exclusively one of $A_1$ or $B_2$. Furthermore, for any $a\in A_1$, there is exactly one $n\in \N$ satisfying $f(n) = a$, and there is exactly one $m\in N$ satisfying $h(m) = f(n)$ given by $m=2n-1$. Similarly, for every $b\in B_2$, there is exactly one $n$ satisfying $g(n)=a=h(2n)$. Therefore, $h$ is a 1-1 correspondence. Otherwise, if $B_2$ is finite, then it has some $k$ elements and there exists a 1-1 correspondence $g:\{1,2,\ldots,k\} \to B_2$. Then, we may define $h = \begin{cases} g(n)& n \leq k \\ f(n-k) &n > k\end{cases}$. Now, for every $a\in A_1$ there is exactly one $n\in \N$ and $m\in\N$ satisfying $f(n)=a=h(m)$ for $m=n+k$. Moreover, for any $b \in B_2$, $f(n)=b=h(n)$ for exactly one $n\in \N$. Therefore, $h$ is once again a 1-1 correspondence. The union of any two countable sets is countable. The general statement for the union of $m$ sets follows by induction. We have that $A_1$ and $A_2$ being countable implies $A_1 \cup A_2$ is countable. Now assume $\bigcup_{n=1}^m A_n$ be countable, then $\bigcup_{n=1}^m A_n \cup A_{n+1}$ is the union of countable sets, and thus itself countable. 
    
    \item Induction cannot be used to show that $\bigcup_{n=1}^\infty A_n$ is countable, because induction only proves the statement for the union of any finite number of sets. There is no $m\in \N$ where $m = \infty$.

    \item Let $A_1, A_2, \dots$ all be countable, meaning $B_m = A_m \backslash \bigcup_{n=1}^{m-1} A_n$ is also countable. Therefore, we can enumerate $B_m$ so that $B_m = \{ b_{m,n} : n\in N \}$. Then, we may lay all the elements of $B_m$ along the $m^\text{th}$ row of a 2-dimensional array, something like

    \begin{center}
    \begin{tabular}{cccc}
        $b_{1,1}$ & $b_{1,2}$ & $b_{1,3}$ & \ldots \\
        $b_{2,1}$ & $b_{2,2}$ & \ldots \\ 
        $b_{3,1}$ & \ldots \\
        \vdots
    \end{tabular}
    \end{center}
    
    Now, for every $a \in \bigcup_{n=1}^\infty A_n$, if $a \in A_m$, then $a \in B_m$ if and only if $a\notin A_k$ for all $k<m$. Since the sequence of sets $A_1,A_2,\ldots$ is countable, the subset of them additionally containing $a$ is also countable. Therefore, if $A_m$ is the first qualifying set (i.e. $f(1) = A_m$, where $f:\N \to \{A_n : a\in A_n, n\in N\}$), then $a\in B_m$ and $a\notin B_k$ for all $k \neq m$. And so, every $a \in \bigcup_{n=1}^\infty A_n$ has exactly one position in the grid.
    
    Now we design a similar grid for the natural numbers, something like
    
    \begin{center}
    \begin{tabular}{cccc}
        1 & 3 & 6 & \ldots \\
        2 & 5 & \ldots \\ 
        4 & \ldots \\
        \vdots
    \end{tabular}
    \end{center}

    We can see that each natural has exactly one position in the grid. So, we may define $f: \N \to \bigcup_{n=1}^\infty A_n$ where $f(n) = b_{i,j}$ where $i$ and $j$ denote the row and column that $n$ appears at     the intersection of in the grid of natural numbers. 
\end{enumerate}

\paragraph{Exercise 1.5.4.}
\begin{enumerate}[(a)]
    \item For any interval $(a,b)$, we can define $f(x) = (2x - a - b)/(b-a)$ which sets $(a,b) \sim (-1,1)$. Thus, since $(-1,1)\sim \R$, we have $(a,b) \sim \R$.
    \item There is probably a number of valid functions proving $(a,\infty) \sim \R$. One example is $f(x) = \ln(x-a)$. For any $y\in R$, we have $\ln(x) = y$ when $x = e^y +a$, so $f$ is onto. Also, $d/dx \ln(x-a) = 1/(x-a)$, which is positive for all $x > a$. Therefore, $f$ is 1-1.
    \item Let $f:(0,1) \to [0,1)$ be given by $f(x) = \inf(x,1)$. Since $0<x<1$, we have that $f(x)$ always a non-empty empty, and so it has an infimum, and moreover that infimum is in $[0,1)$. Less nicely, we may also define the inverse, $f\inv : [0,1) \to (0,1)$ where $f\inv(x)$ is lower bound of the open set for which $x$ is the infimum.
\end{enumerate}

\paragraph{Exercise 1.5.5.}
\begin{enumerate}[(a)]
    \item For any set $A$, let $f: A\to A$ be given by $f(a) = a$. Then every element is mapped to by and only by itself. Therefore $A \sim A$ is always the case.
    \item $A\sim B$ means there exists a 1-1 correspondence $f: A\to B$. Let $g:B\to A$ be given by $g(b) = \{ a\in A : f(a) = b\}$. Since $f$ is 1-1, whenever $f(a_1) = f(a_2) = b$, it must be that $a_1 = a_2$, and so $g(b)$ always maps to exactly one element in $A$. Also, if $g(b_1) = g(b_2)$, that means $f(a)=b_1$ and $f(a)=b_2$ for some $a\in A$, which is only possible if $b_1 = b_2$. Therefore $g$ is 1-1. Additionally, for any $a \in A$, $g(b)=a$ for $b$ satisfying $f(a)=b$, which always exists since $f$ is onto. Therefore $g$ is a 1-1 correspondence, and we may conclude that $A\sim B$ if and only if $B \sim A$.
    \item If $A\sim B$ and $B\sim C$ then there exists 1-1 onto functions $f: A\to B$, and $g: B\to C$. Let $h:A\to C$ be given by $h(a) = g(f(a))$ for any $a \in A$. Let $c$ be an arbitrary element in $C$, then there is exactly one $b\in B$ satisfying $g(b) = c$, and similarly exactly one $a\in A$ satisfying $f(a) = b$. And so, that makes exactly one $a\in A$ satisfying $h(a) = g(f(a)) = g(b) = c$. Therefore, $h$ is 1-1 and onto, and we may conclude that if $A\sim B$ and $B\sim C$ then $A\sim C$.
\end{enumerate}

\paragraph{Exercise 1.5.6.}
\begin{enumerate}[(a)]
    \item Take the collection of disjoint open intervals $\{ (1,2), (2,3), (3,4) \ldots \}$, where the $n^{th}$ interval is given by $(n,n+1)$.
    \item \textit{Solved with hint.} Assume for the sake of contradiction that there exists some uncountable collection of disjoint open intervals. Then, each interval has a real lower and upper bound, and since $\Q$ is dense in $\R$, we may find a unique rational number in each interval. That would mean $\Q$ is uncountable, which is not true, and thus a contradiction.
\end{enumerate}

\paragraph{Exercise 1.5.7.}
\begin{enumerate}[(a)]
    \item Let $f:(0,1) \to S$ be given by $f(x) = (0,x)$.
    \item Let $f:S \to (0,1)$ be given by $f(x,y) = 0.x0y$. For example, $f(0.1,0.223) = 0.10223$. This function in 1-1 but not onto. For example, there is no element of $S$ that maps to $0.123$ under $f$ because there is no 0.
\end{enumerate}

\paragraph{Exercise 1.5.8.}
\textit{Solved with \href{https://math.stackexchange.com/a/2876333/1027565}{hint}}. Assume there were some $x\in B$ with $x>2$. Then, $\{x\}$ is a finite subset of $B$ with sum greater than $2$, which would violate the premise. Therefore, $B \subseteq (0,2]$. Furthermore, if $2\in B$, then $B = \{2\}$, because otherwise there would be some other $x\in B$ violating $x+2 \leq 2$. If $2\notin B$, then it follows that $B\subseteq (0,2)$. In that case, we may write $B \subseteq \bigcup_{n=2}^\infty [\frac{2}{n},\frac{2}{n-1})$. To prove it, first note that $\bigcup_{n=2}^{m=2} [\frac{2}{n},\frac{2}{n-1}) = [1, 2) = [2/m, 2)$. Then, assume $\bigcup_{n=2}^m [\frac{2}{n},\frac{2}{n-1}) = [2/m, 2)$, and take the union of both sides with $[2/(m+1), 2m)$ to get $\bigcup_{n=2}^{m+1} [\frac{2}{n},\frac{2}{n-1}) = [2/(m+1), 2)$. Now, let $x$ be an arbitrary element of $b$, then there exists $n\in \N$ so that $1/n < x/2$ and equivalently $2/n < x$. Then certainly $x\in [2/n, 2) = \bigcup_{k=2}^n [2/k, 2/(k-1)) \subseteq \bigcup_{k=2}^\infty [2/k, 2/(k-1))$. Thus proving that $B \subseteq \bigcup_{n=2}^\infty [\frac{2}{n},\frac{2}{n-1})$. Importantly, for any $n\in N$, it must be the case that $[2/n, 2/(n+1))$ has at most $n$ elements. To see why, assume for the sake of contradiction that the interval has finite $k>n$ elements. Then the sum of those elements is surely at least $k\times 2/n$, as $2/n$ is the smallest element. But $k\time 2/n > 2$ for $k>n$, which violates our premise. If each interval cannot contain more than a finite number of elements from $B$, it surely cannot contain an infinite amount. And so, we have $B$ as a subset of an infinite sum of countable sets, which makes it then either countable or finite.

\paragraph{Exercise 1.5.9.}
\begin{enumerate}[(a)]
    \item $(\sqrt{2})^3 - 2(\sqrt{2}) = 0$. 
    
    $(\sqrt[3]{2})^3 - 2 = 0$. 
    
    $(\sqrt{3})^4 + (\sqrt{2})^2 - 11 = 0$.
    \item The set of possible values for the $k^{th}$ integer coefficient of a polynomial may be enumerated as $\{n^k_0, n^k_1, \ldots \}$.  Then, for a polynomial of degree $k$, let $P_{n,k}$ be the set of all coefficient orderings who's sum is $n$ when mapped to the natural numbers. So $P_{n,k} = \{(n^k, n^{k-1}, \ldots, n^0) : n^k + n^{k-1} + \dots + n^0 = n\}$. As a lenient bound, $1 < n^m \leq n$, and so $P_{n,k}$ is finite. Moreover, let $C_k$ be the set of all coefficient ordering of degree $k$. Then $C_k = \bigcup_{n=1}^\infty P_{n,k}$, which means $C_k$ is countable. Let $c^k_n$ be the $n^{th}$ coefficient ordering of a polynomial of degree $k$, and let $R(c^k_n)$ be the set of it's roots. We have that $R(c^k_n)$ is always finite, and thus $\bigcup_{n=1}^\infty R(c^k_n) = A_k$ is countable.
    \item Every algebraic number is in some $A_k$, and every element of any $A_k$ is and algebraic number, and so the set of all algebraic number is $\bigcup_{k=1}^\infty A_k$. Therefore, there is a countable number of algebraic numbers. As a result, there are an uncountable number of transcendental numbers, because otherwise $\R$ would be countable.
\end{enumerate}

\paragraph{Exercise 1.5.10.}
\begin{enumerate}[(a)]
    \item We may write the interval $[0,1]$ as $\{0\} \cup \bigcup_{n=1}^\infty [1/n,1]$. Thus, $C = C \cap [0,1] = C \cap \left(\{0\} \cup \bigcup_{n=1}^\infty [1/n,1]\right) = (C\cap\{0\}) \cup \bigcup_{n=1}^\infty(C \cap [1/n,1])$. Surely $C\cap \{0\}$ is finite. Now assume for the sake of contradiction that $C\cap [a,1]$ is countable for all $a\in(0,1)$. For any $n$, it follows that $1/n \in (0,1)$ and thus $C\cap [1/n,1]$ is countable. That would make $C$ a union of countable sets, and thus countable. Since $C$ is supposed to be uncountable, it must be that there exists some $a\in(0,1)$ such that $C\cap[a,1]$ is uncountable.
    \item If $\alpha = 1$ then $C \cap [\alpha,1]=C\cap\{1\}$ is finite. Surely $\alpha > 0$, because by definition $\alpha \geq a$ for all $a\in A$, and $a\subseteq (0,1)$. Thus, if $\alpha < 1$, then $\alpha \in (0,1)$. In that case, assume for the sake of contradiction that $C\cap[\alpha,1]$ is uncountable. Then $\alpha=\max A$, and so for any $[a,1]$ where $a>\alpha$, it must be that $C\cap[a,1]$ is countable, because $a \notin A$. It then follows that $C\cap(\alpha,1]$ must be countable. If that is true, then $(C\cap \{\alpha\})\cup (C\cap(\alpha,1])$ must be countable, as it is a union of finite and countable sets. Moreover, that set is the same as $C\cap(\{\alpha\}\cup(\alpha,1]) = C \cap [\alpha, 1]$. But there lies a contradiction, as we supposed that $C \cap [\alpha, 1]$ was uncountable. Therefore, $C \cap [\alpha, 1]$ must be countable.
    \item \textit{Counter-example.} Let $C\subseteq [0,1]$ be given by $\{1/n: n\in \N\}$. Written out, $C$ looks something like 
    $\{\ldots, \frac{1}{4}, \frac{1}{3}, \frac{1}{2}, 1\}$.
    Now for any $a\in(0,1)$ there exists $n \in \N$ satisfying $n < \frac{1}{a}$, or rather $\frac{1}{n}>a$. As a result, 
    $C \cap [a,1] \subseteq C \cap [\frac{1}{n},1]$. Note that $C\cap [\frac{1}{n},1] = \{\frac{1}{n}, \frac{1}{(n-1)}, \ldots, 1\}$ which is finite with exactly $n$ elements. Therefore, $C\cap[a,1]$ must also be finite for any $a\in(0,1)$.
\end{enumerate}

\paragraph{Exercise 1.5.11 (Schroder–Bernstein Theorem).}
\begin{enumerate}[(a)]
    \item We may then define $h(a) = \begin{cases}
        f(a) & a\in A \\
        g\inv(a) & a\in A'
    \end{cases}$ for all $a\in A$. Since $h$ is a 1-1 map from $A$ onto $B$, we may conclude that $A\sim B$.
    \item If $A_1 = \emptyset$, then $g$ is onto and so we have our 1-1 correspondence. Assume for the sake of contradiction that there exists $m>n$ where $A_m\cap A_n$ is not empty. That means $(g\circ f)^m(x_0) = (g\circ f)^n(x_1)$ for some $x_0$ and $x_1$ in $X$. Since $g$ and $f$ are 1-1, we may iteratively take the inverse of each to get $(g\circ f)^{m-n}(x_0)=x_1$. Observing that the left side is of the form $g(y)=x_1$ for some $y \in Y$, it follows that $x_1 \in g(Y)$. Therefore, $x_1 \notin A_1$, but that would imply that $(g \circ f)^n(x_1)\notin A_n$, which is a contradiction. And so, we must admit that $A_m \cap A_n = \emptyset$ for all $m \neq n$. Moreover, assume for the sake of contradiction that $f(A_m)$ and $f(A_n)$ are not disjoint for some $m \neq n$, that is, they share some $y \in Y$. Let $x_0 \in A_m$ and $x_1 \in A_n$ satisfy $f(x_0)=y=f(x_1)$. Since $f$ is onto, it must be that $x_0 = x_1$, which contradicts the fact that $A_m$ and $A_n$ are themselves disjoint. Therefore, $f(A_m) \cap f(A_n) = \emptyset$.

    \item $f(A) = \{f(x):x\in A\} = \{f(x):x\in A_1 \cup A_2 \cup \ldots \} = \{f(x):x\in A_1\} \cup \{f(x): x\in A_2\} \cup \ldots = f(A_1)\cup f(A_2) \cup \ldots = \bigcup_{n=1}^\infty f(A_n) = B$. And so, if $f$ did not map $A$ onto $B$, then there would have to exists some $b\in B$ for which no $a\in A$ satisfied $f(a)=b$. In that case, $b \notin f(A) = B$, which is contradiction. So, $f$ does map $A$ onto $B$.

    \item If $A'$ is empty, any function maps onto it. So let $A'$ be non-empty. An arbitrary element $a$ is in $A'$ if and only if $a\notin A$. That is, $a \notin \bigcup_{n=1}^\infty A_n$, and by De Morgan's law, $a\in \bigcap_{n=1}^\infty A_n^c$. Certainly then, $a \notin A_1$, which means $a\in g(Y)$, and so $g\inv(a) \in Y$. Note that, for $n>1$, we have $a \in A_n^c$ if and only if $a\notin g(f(A_{n-1}))$, that is, $g\inv(a)\notin f(A_{n-1})$. Therefore, $g\inv(a) \in \left( \bigcup_{n=1}^\infty f(A_n) \right)^c = B'$, and so $g$ maps $B'$ onto $A'$.
\end{enumerate}

\subsection{Cantor's Theorem}
\paragraph{Exercise 1.6.1.}
From a previous exercise we found that $(0,1) \sim \R$, and also that $\sim$ is a transitive operator. So, if $(0,1)$ is countable, then $\N \sim (0,1)$, and so $\N \sim \R$, meaning $\R$ is countable. Similarly, if $\R$ is countable then $(0,1)$ is countable. Equivalently, $(0,1)$ is uncountable if and only if $\R$ is uncountable.

\paragraph{Exercise 1.6.2.}
\begin{enumerate}[(a)]
    \item 
	The first digit of $x$, given by $b_1$, equals $2$ if $a_{11} \neq 2$ and $3$ otherwise. 
	Therefore, $b_1\neq a_{11}$, and as results, $x \neq f(1)$.

    \item 
	The second decimal digit of $x$ is $b_2$, and $b_2\neq a_{22}$, which is the second digit of $f(2)$. 
	Therefore, $x\neq f(2)$. 
	In general, for any $n\in \N$, the $n^{th}$ digit of x, given by $b_n$, satisfies $b_n\neq a_{nn}$, and so $x\neq f(n)$. 

    \item 
	We have that, for any $n$, it follows that $x \neq f(n)$.
	So, assume, for the sake of contradiction, that $x$ has been enumerated, that is, $f(n) = x$ for some $n$.
	Then, it follows that $x \neq f(n) = x$.
	That is a contradiction, and so it must be that $x$ was never enumerated, and so no $n$ maps to $x$ under $f$.
	Therefore, we cannot construct a map from $\N$ onto $(0,1)$, proving that it is uncountable.
\end{enumerate}

\paragraph{Exercise 1.6.3.}
\begin{enumerate}[(a)]
    \item 
	We guarantee only that $x=0.b_1b_2\ldots$ is real number, which poses no contradiction with a supposed enumeration of the rationals.

    \item 
	Assume that $f$ maps natural numbers to the terminating decimal expansion of every real number.
	Such an assumption is fine, because if $f$ really is 1-1 and onto, then it remains so for the real numbers.
	Then, the diagonally constructed element, $x$, will also be a terminating expansion, as it contains no $9$s.
	So, there is no possibility of $x$ being the non-terminating equivalent of an enumerated number.
	Therefore, $x$ remains distinct from every enumerated number.
\end{enumerate}

\paragraph{Exercise 1.6.4.}
Every sequence has a well defined $n^{th}$ digit, and so each sequence is countable. Now, assume for the sake of contradiction that the set of all binary sequences, $S$, is itself countable. Then, we must be able to enumerate the sequences in a grid like so
\begin{center}
    \begin{tabular}{cccc}
         $a_{11}$ & $a_{12}$ & $a_{13}$ & \ldots \\
         $a_{21}$ & $a_{22}$ & \ldots \\
         $a_{31}$ & \ldots \\
         \vdots
    \end{tabular}
\end{center}
Now, define the sequence $x = (b_1,b_2,\ldots)$ where $b_n = \begin{cases}
    0 & a_{nn}=1\\
    1 & a_{nn}=0
\end{cases}$.
Then, for any $n\in \N$ the sequence at row $n$ must differ from $x$ in the $n^{th}$ digit. Moreover, $x$ is just a sequence of $0$s and $1$s, and so it is a valid member of $S$. Therefore, $x\in S$ has not been enumerated, and so $S$ must be uncountable.

\paragraph{Exercise 1.6.5.}
\begin{enumerate}[(a)]
    \item $P(A)=\{\emptyset,\{a\},\{b\},\{c\},\{ab\},\{ac\},\{bc\},\{abc\}\}$
    \item For any $a\in A$ and $p\in P(A)$, either $a\in p$ or $a\notin p$. 
    So, if $A$ has $n$ elements, then there are $2^n$ ways to construct a valid set for $p(A)$.
\end{enumerate}

\paragraph{Exercise 1.6.6.}
\begin{enumerate}[(a)]
    \item Let $f(a)=\{a\}$, $f(b)=\{b\}$, $f(c)=\{c\}$. Or another: $f(a)=\{abc\},f(b)=\emptyset,f(c)=\{bc\}$.
    \item Let $g(1)=\{1\}$, $g(2)=\{2\}$, \ldots
    \item We cannot construct an onto mapping from $A$ to $P(A)$ because $P(A)$ has more elements.
\end{enumerate}

\paragraph{Exercise 1.6.7.}
\begin{enumerate}[(a)]
    \item $B=\emptyset$. $B=\{b\}$
    \item $B=\emptyset$
\end{enumerate}

\paragraph{Exercise 1.6.8.}
\begin{enumerate}[(a)]
    \item We assume that $f(a')=B$. If $a'\in B$, then $a'\in f(a')$, and thus $a'\notin B$.
    \item Conversely, if $a'\notin B$, then $a'\notin f(a')$, and thus $a'\in B$.
\end{enumerate}

\paragraph{Exercise 1.6.9.}
To show that $P(\N)\sim\R$ it will suffice to show that $P(\N)\sim(0,1)$. For any $x\in (0,1)$, where the decimal expansion of $x=.a_1a_2a_3\ldots$, let $nd$ (where $n$ and $d$ are concatenated) be an element of $f(x)$ if and only if $a_n=d$. For example $f(0.021)=\{10,22,31,40,50,\ldots\}$. To verify that $f(x)\in P(\N)$ for any $x$, note that $n\in \N$ and $d\in \{0,1,\ldots,9\}$, and so $nd$ is just a natural number with a digit added to the right. So, any element of $f(x)$ is natural number, meaning $f(x)$ is a set of naturals and therefore in $P(\N)$. To verify that $f$ is 1-1, we take for granted that if two elements of $(0,1)$ have the exact same decimal expansion, then they must be the same number. Conversely, if they are different, then they differ at some digit. Let $x=.a_1a_2\ldots$ differ from $y=b_1b_2\ldots$ at the $n^{th}$ digit, so that $a_n\neq b_n$. Then, by the definition of $f$, it follows that $a_n\in f(x)$ but $a_n \notin f(y)$ so $f(x)\neq f(y)$. Therefore, $f$ is 1-1. 

Now we construct a similar sort of function from $P(\N)$. Let $S\in P(\N)$, and let $a_n$ be the $n^{th}$ decimal of $g(S)$. Then define $a_n=\begin{cases}
    1 & n\in S \\
    0 & n\notin S
\end{cases}$, and in the special case that $S=\emptyset$, let $g(S)=0.2$.
For example, $g(\{1,4,5\})=.10011000\ldots$ and so on. To verify that $g$ maps into $(0,1)$, note that $0<g(S) \leq 0.2$. To verify that $g$ is 1-1, let there be some $n\in S_1$ where $n \notin S_2$ so that $S_1\neq S_2$. Then the $n^{th}$ digit of $g(S_1)=1$ while the corresponding digit in $g(S_2)=0$, thus they have a differing decimal expansion, and so it follows that $g(S_1)\neq g(S_2)$.

By now we have a 1-1 function $f:(0,1)\to P(\N)$ and a 1-1 function $g:P(\N)\to (0,1)$. So, we may use the Bernstein-Schroeder theorem to conclude that $P(\N)\sim (0,1)$, and as a result, $P(\N)\sim\R$.

\paragraph{Exercise 1.6.10.}
\begin{enumerate}[(a)]
    \item
	We may represent a function $f: \{0,1\}\to \N$ as a pair of natural numbers $(n_0, n_1)$ where $f(0)=n_0$ and $f(1)=n_1$.
	Let $A_k$ be the set of all pairs where $n_0 + n_1 = k$.
	Every $(n_0, n_1)$ sums to exactly one number, and thus appears only in $A_{k}$ where $k=n_0 + n_1$.
	Moreover, each $A_k$ is finite, and so the union of all such $A_k$, that is, $\bigcup_{n=1}^\infty A_k$, is countable.
	Therefore, the set of all such $f$ is countable as well.

	\item 
	A function $f$ from $\N$ to $\{0,1\}$ may be represented as an infinite sequence $(a_1,a_2,a_3,\ldots)$, where $a_n = f(n)$, and vice versa.
	That is, the value of the $n^{th}$ digit of the sequence indicates which element of the set, $\{0,1\}$, that $f(n)$ maps to.
	Any $f$ may be represented as a valid binary sequence, and every binary sequence is the representation of exactly one $f$.
	To see why, let $S_1=(a_{11}, a_{12}, \ldots\}$ and $S_2$ be different binary sequences.
	Then there is some $n$ such that the $n^{th}$ element of 
	We determined in Exercise 1.6.4 that the set of all such sequences is uncountable, and thus the set of all such $f$ is as well.

	\item \textit{Solved with \href{https://math.stackexchange.com/a/4858163/1027565}{hint}}
	To construct an uncountable antichain from $P(\N)$, we first partition $\N$ into sets of two.
	Let $A_n = \{2n, 2n+1\}$, so for example, $A_1 = \{2,3\}, A_2 = \{4, 5\}, A_3 = \{6,7\}$ and so on.
	Consider a set constructed by selecting exactly one number from each $A_n$.
	We may represent such a set as the binary sequence $S=(a_1,a_2,a_3,\ldots)$, where $a_n = 0$ if the smaller element of the $n^{th}$ set is chosen, and $a_n=1$ if the larger is chosen.
	Then, any infinite binary sequence represents a valid sequence of choices, and any two binary sequences that differ must differ at some index.
	So, the two sets represented by different sequences each contain an element that the other does not, and thus are not subsets of eachother.
	Therefore, we may follow the uncountability arugment for inifinte binary sequences, and conclude that we have constructed an uncountable antichain of $P(\N)$.

\end{enumerate}

\end{document}
